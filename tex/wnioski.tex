\section{Wnioski}
    \subsection{Podsumowanie}
    W trakcie realizacji projektu nad katalogiem piłkarzy opartym na OOP w PHP, skupiliśmy się na stworzeniu  systemu, który integruje logiczne obiekty piłkarzy w spójny model danych. Implementacja strony internetowej za pomocą HTML, CSS i JavaScript pozwoliła nam na stworzenie atrakcyjnego i interaktywnego interfejsu dla użytkowników. Wykorzystanie architektury MVC umożliwiło nam lepszą strukturę projektu, dzięki czemu nasza aplikacja jest łatwiejsza w zarządzaniu, rozwijaniu i testowaniu.
    \subsection{Podział pracy}

    U-20019 - Jakub Achtelik\\
    U-20041 - Oliwier Budnik\\

    \subsubsection{Zadania wykonane przez U-20019}
    \begin{itemize}
        \item Zaprojektowanie oraz wykonanie bazy danych
        \item Konstrukcja zapytań oraz klas, które służą do łączenia z bazą danych
        \item Konfiguracja usług na serwerze VPS
        \item Implementacja połączenia API Wikipedia z naszą aplikacją
        \item Proste grafiki w programie GIMP
        \item Pobieranie danych od użytkownika za pomocą POST oraz GET
    \end{itemize}
    \subsubsection{Zadania wykonane przez U-20041}
    \begin{itemize}

        \item Prototyp wyglądu aplikacji
        \item Implentacja autoryzacji oraz formularz logowania
        \item Formularz edycji, dodawania i przekazanie tego do jednej tablicy
        \item Wyświetlanie zwróconych wyników z bazy danych w formie Szablonów HTML
        \item Szablon strony HTML oraz dostosowanie wyglądu w CSS
        \item Routing strony, czyli ustawienie klasy, która na podstawie /linku wykonuje określoną operację
        \item Implentacja logiki klasy Aplikacja
    \end{itemize}
